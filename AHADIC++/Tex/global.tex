%= math ===========================================
\usepackage{amsmath}
\usepackage{amssymb}
\usepackage[a4paper,pdfborder={0 0 0}]{hyperref}
\usepackage[all]{hypcap}
\usepackage{array}
\usepackage{calc}
\usepackage{longtable}
\usepackage{multirow}
%= graphics =======================================
\usepackage{axodraw}
\usepackage{pstricks}
\usepackage{graphicx}
\usepackage{xspace}
%\usepackage{graphpap}
%= citations ======================================
\usepackage{cite}
\usepackage{mcite}
%= layout =========================================
\usepackage[format=hang,labelfont=bf]{caption}
\usepackage{sectsty}
\allsectionsfont{\sffamily}
\subsubsectionfont{\mdseries\itshape\large}
\setlength{\parindent}{0mm}
%\setlength{\voffset}{-1cm}
\setlength{\hoffset}{-1.75cm}
\setlength{\textwidth}{16.5truecm}
\setlength{\textheight}{24cm}
\setlength{\topmargin}{0mm}
\setlength{\headheight}{0mm}
\setlength{\headsep}{0mm}
\setlength{\parskip}{1mm}
\setlength{\mathindent}{2ex}
\let\spreprint\empty
\newcommand{\preprint}[1]{\def\spreprint{\protect#1}}
\let\sinstitute\empty
\newcommand{\institute}[1]{\def\sinstitute{\protect#1}}
\makeatletter
\renewcommand{\maketitle}{\begingroup
  \null\thispagestyle{empty}%
    \ifx\spreprint\empty
      \vskip 5ex
    \else
      \flushright\large\spreprint\vskip 2ex
    \fi
    \vskip 5ex
    \flushleft
      {\sffamily\bfseries\huge\@title}\vskip 2ex
      \@author\vskip 2ex
      \ifx\sinstitute\empty
      \else
        {\small\sinstitute}
      \fi
    \vskip 5ex
  \endgroup
}
\makeatother
\renewenvironment{abstract}{\begin{center}
  {\large\sffamily\bfseries Abstract: }
  \begin{minipage}[t]{0.75\textwidth}
}{\end{minipage}\end{center}\vskip 10ex}
\newenvironment{stress}{\vskip 2ex
  \hspace*{2ex}\begin{minipage}{\textwidth-4ex}
  \em}{\end{minipage}\vskip 2ex}
\newcommand{\myfigure}[3]{
  \begin{figure}[#1]
    \begin{center}
      #2\\
      \parbox[t]{\widthof{#2}}{\caption{#3}}
    \end{center}
  \end{figure}
}
\newcommand{\mytable}[3]{
  \begin{table}[#1]
    \begin{center}
      #2\\
      \parbox[t]{\widthof{#2}}{\caption{#3}}
    \end{center}
  \end{table}
}
%= abbreviations ==================================
\newcommand{\MCatNLO}{M\protect\scalebox{0.8}{C}@N\protect\scalebox{0.8}{LO}\xspace}
\newcommand{\HERWIG}{H\protect\scalebox{0.8}{ERWIG}\xspace}
\newcommand{\HERWIGpp}{H\protect\scalebox{0.8}{ERWIG++}\xspace}
\newcommand{\Ariadne}{A\protect\scalebox{0.8}{RIADNE}\xspace}
\newcommand{\POWHEG}{P\protect\scalebox{0.8}{OWHEG}\xspace}
\newcommand{\Sherpa}{S\protect\scalebox{0.8}{HERPA}\xspace}
\newcommand{\Comix}{C\protect\scalebox{0.8}{OMIX}\xspace}
\newcommand{\Apacic}{A\protect\scalebox{0.8}{PACIC++}\xspace}
\newcommand{\Amegic}{A\protect\scalebox{0.8}{MEGIC++}\xspace}
\newcommand{\Rivet}{R\protect\scalebox{0.8}{ivet}\xspace}
\newcommand{\Professor}{P\protect\scalebox{0.8}{rofessor}\xspace}
\newcommand{\FeynRules}{F\protect\scalebox{0.8}{EYN}R\protect\scalebox{0.8}{ULES}\xspace}
\newcommand{\CSS}{C\protect\scalebox{0.8}{SS}\xspace}
\newcommand{\Ahadic}{A\protect\scalebox{0.8}{HADIC++}\xspace}
\newcommand{\Hadrons}{H\protect\scalebox{0.8}{ADRONS++}\xspace}
\newcommand{\Photons}{P\protect\scalebox{0.8}{HOTONS++}\xspace}
\newcommand{\Pythia}{P\protect\scalebox{0.8}{YTHIA}\xspace}
\newcommand{\Jetset}{J\protect\scalebox{0.8}{ETSET}\xspace}
%= definitions ====================================
\long\def\symbolfootnote[#1]#2{\begingroup%
\def\thefootnote{\fnsymbol{footnote}}\footnote[#1]{#2}\endgroup}
\newcommand{\abs}[1]{\left| #1\right|}
\newcommand{\rbr}[1]{\left( #1\right)}
\newcommand{\abr}[1]{\langle #1\rangle}
\newcommand{\cbr}[1]{\left\{ #1\right\}}
\newcommand{\sbr}[1]{\left[ #1\right]}
\newcommand{\beq}{\begin{equation}}
\newcommand{\eeq}{\end{equation}}
\newcommand{\bal}{\begin{align}}
\newcommand{\eal}{\end{align}}
\newcommand{\done}{{\rm d}}
\newcommand{\dtwo}{{\rm d}^2}
\newcommand{\dthree}{{\rm d}^3}
\newcommand{\dfour}{{\rm d}^4}
\newcommand{\order}{\mathcal{O}}
\newcommand{\mc}[1]{\mathcal{#1}}
\newcommand{\DO}{D$0\!\!\!/$ }
\newcommand{\dst}{\displaystyle}
\newcommand{\sst}{\scriptstyle}
\newcommand{\qcut}{Q_{\mathrm{cut}}}
\newcommand{\Nmax}{N_{\mathrm{max}}}
\newcommand{\GeV}{\mathrm{GeV}}
\newcommand{\bV}{{\bf V}}
\newcommand{\bT}{{\bf T}}
\newcommand{\CF}{C_{\mathrm{F}}}
\newcommand{\Nc}{N_{\mathrm{c}}}
\newcommand{\CA}{C_{\mathrm{A}}}
\newcommand{\TR}{T_{\mathrm{R}}}                                               
\newcommand{\q}{\mathrm{q}}
\newcommand{\Q}{\mathrm{Q}}
\newcommand{\qbar}{\mathrm{\overline{q}}}
\newcommand{\Qbar}{\mathrm{\overline{Q}}}
\newcommand{\g}{\mathrm{g}}
\newcommand{\kperp}{k_\perp}
\newcommand{\kperpbf}{{\bf{k}}_\perp}
\newcommand{\kperpbfsq}{{\bf{k}}_{\perp}^{\,\!2}}
\newcommand{\kperpzero}{{\bf{k}}_{\perp,0}}
\newcommand{\kperpzerosq}{{\bf{k}}_{\perp,0}^{\,\!2}}
\newcommand{\kperpmax}{{\bf{k}}_{\perp,{\rm max}}}
\newcommand{\logyc}{\log_{10} (\qcut^{2}/s)}
%==================================================
